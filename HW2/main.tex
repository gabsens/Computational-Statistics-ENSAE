\documentclass[a4paper,11pt]{article}

\usepackage{listings}
\usepackage{lmodern}
\usepackage{amsmath,amssymb,amsthm,textcomp}
\usepackage[utf8]{inputenc}  
\usepackage[T1]{fontenc}  
\usepackage{ listings }
\usepackage{graphicx}
\usepackage{float}
\usepackage{enumitem}
\usepackage{bbm}
\usepackage{caption}
\graphicspath{ {images/} }

\usepackage{geometry}
\geometry{total={210mm,297mm},
left=15mm,right=15mm,%
bindingoffset=0mm, top=20mm,bottom=20mm}

\usepackage{etoolbox}
\makeatletter
\preto{\@verbatim}{\topsep=0pt \partopsep=3pt }
\makeatother

\linespread{1}

\newcommand{\linia}{\rule{\linewidth}{0.5pt}}


% my own titles
\makeatletter
\renewcommand{\maketitle}{
\begin{center}
\vspace{2ex}
{\huge \textsc{\@title}}
\vspace{1ex}
\\
\linia\\
\@author 
\vspace{4ex}
\end{center}
}
\makeatother
%%%

% custom footers and headers
\usepackage{fancyhdr}
\pagestyle{fancy}
\lhead{}
\chead{}
\rhead{}

\cfoot{}
\rfoot{Page \thepage}
\renewcommand{\headrulewidth}{0pt}
\renewcommand{\footrulewidth}{0pt}
%
\DeclareMathOperator*{\argmax}{arg\,max}
\DeclareMathOperator*{\argmin}{arg\,min}
\DeclareMathOperator*{\var}{var}
\DeclareMathOperator*{\cov}{cov}

%%%----------%%%----------%%%----------%%%----------%%%

\begin{document}

\title{HW2}

\author{Gabriel ROMON}



\maketitle

\section*{Problem 4.15}

\begin{enumerate}[label=(\alph*)]
  \item Since $\delta_2 = \delta_1 + \beta(\delta_3-P(X>\mu))$ and $P(X>\mu)$ is a constant, $\begin{aligned}[t]\var(\delta_2) &= \var(\delta_1)+\var(\beta \delta_3) + 2\cov(\delta_1,\beta \delta_3)\\
  &= \var(\delta_1)+\beta \var( \delta_3) + 2\beta\cov(\delta_1, \delta_3)
  \end{aligned}$. Since the $X_i$ are i.i.d, $\var(\delta_3) = \frac 1{n^2} \cdot n \cdot \var(\mathbbm 1_{X>\mu}) = \frac 1n P(X>\mu)(1-P(X>\mu))$. Besides, $$\begin{aligned} 
  \cov(\delta_1, \delta_3) &= \cov\left(\frac 1n \sum_{i=1}^n \mathbbm 1_{X_i> \alpha}, \frac 1n \sum_{j=1}^n \mathbbm 1_{X_j> \mu} \right)
  = \frac 1{n^2} \sum_{1\leq i,j\leq n} \cov(\mathbbm 1_{X_i> \alpha}, \mathbbm 1_{X_j> \mu})\\
  &= \frac 1{n^2} \sum_{i=1}^n \cov(\mathbbm 1_{X_i> \alpha}, \mathbbm 1_{X_i> \mu}) \quad \text{by independence}\\
  &= \frac 1{n^2} \cdot n \cdot \left[E(\mathbbm 1_{X> \alpha} \mathbbm 1_{X> \mu}) - E(\mathbbm 1_{X> \alpha}) E(\mathbbm 1_{X> \mu})\right]\\
  &= \frac 1n P(X>a)(1-P(X>\mu))
  \end{aligned}$$

  \item Since $\var(\delta_2) = \var(\delta_1)+\beta \var( \delta_3) + 2\beta\cov(\delta_1, \delta_3)$, $\delta_2$ improves over $\delta_1$ in terms of variance iff $\beta^2\var(\delta_3) + 2\beta \cov(\delta_1, \delta_3)<0$. Since $\frac{\cov(\delta_1, \delta_3)}{\var(\delta_3)} = \frac{P(X>a)}{P(X>\mu)}>0$, this can only happen iff $\beta<0$ and $\beta > -2 \frac{\cov(\delta_1, \delta_3)}{\var(\delta_3)} = -2\frac{P(X>a)}{P(X>\mu)}$.\newline \newline
  In order to find a suitable $\beta$, one has to know a lower bound on $P(X>a)$. Note that the optimal $\beta$ remains unknown.

  \item In the normal case, deriving a lower bound on $P(X>a)$ can be done through repeated integration by parts: $$\begin{aligned}
    P(X>a)&= \int_a^\infty \frac{e^{-t^2/2}}{\sqrt{2 \pi}} dt = \int_a^\infty \frac 1t\cdot \frac{te^{-t^2/2}}{\sqrt{2 \pi}} dt = \frac{e^{-a^2/2}}{\sqrt{2 \pi}a} - \int_a^\infty \frac{1}{t^2} \frac{e^{-t^2/2}}{\sqrt{2 \pi}} dt\\
    &= \frac{e^{-a^2/2}}{\sqrt{2 \pi}a} - \left(\frac{e^{-a^2/2}}{\sqrt{2 \pi}a^3} -  \int_a^\infty \frac{3}{t^4} \frac{e^{-t^2/2}}{\sqrt{2 \pi}} dt\right)\\
    &\geq \frac{e^{-a^2/2}}{\sqrt{2 \pi}}\left(\frac 1a -\frac 1{a^3} \right)
  \end{aligned}$$
  Since the normal distribution is symmetric, we set $\mu = 0$ and our value for $\beta$ is therefore $\displaystyle -4\frac{e^{-a^2/2}}{\sqrt{2 \pi}}\left(\frac 1a -\frac 1{a^3} \right)$.\newline
  Python code for simulations is in the Appendix. For $a\in\{3,5,7\}$, we sample $10^5$ points from the normal distribution, compute $\delta_1$ and $\delta_2$ and collect the values of $(\delta_1-P(X>a))^2$ and $(\delta_2-P(X>a))^2$. We repeat this experiment $10^4$ times and average the values we found to get an estimate of the variance of each estimator. The results in the table below show that $\delta_2$ provides indeed a slight improvement.
\end{enumerate}

{\centering
\begin{tabular}{|c||c|c|c|c|} \hline
$a$ &  $P(X>a)$ & Lower bound on $P(X>a)$& Mean of $(\delta_1-P(X>a))^2$ & Mean of $(\delta_2-P(X>a))^2$ \\ \hline
3 &
\begin{tabular}{c} $1.3499 \cdot 10^{-3}$
\end{tabular} &
\begin{tabular}{c} $1.3131 \cdot 10^{-3}$
\end{tabular} &
\begin{tabular}{c} $1.5195 \cdot 10^{-8}$
\end{tabular} &
\begin{tabular}{c} $1.5183 \cdot 10^{-8}$
\end{tabular}  \\ \hline
5 &
\begin{tabular}{c} $2.8665 \cdot 10^{-7}$
\end{tabular} &
\begin{tabular}{c} $2.8545 \cdot 10^{-7}$
\end{tabular} &
\begin{tabular}{c} $2.8726 \cdot 10^{-12}$
\end{tabular} &
\begin{tabular}{c} $2.8725 \cdot 10^{-12}$
\end{tabular}  \\ \hline
7&
\begin{tabular}{c} $1.2799 \cdot 10^{-12}$
\end{tabular} &
\begin{tabular}{c} $1.2783 \cdot 10^{-12}$
\end{tabular} &
\begin{tabular}{c} $1.6381 \cdot 10^{-24}$
\end{tabular} &
\begin{tabular}{c} $1.6380 \cdot 10^{-24}$
\end{tabular}  \\ \hline
\end{tabular}
\captionof{table}{Simulation results for $f\sim \mathcal N(0,1)$}
}

\begin{enumerate}[label=(\alph*)]
  \setcounter{enumi}{3}
  \item It is known that Student's distribution $\mathcal T_{\nu}$ converges in distribution to $\mathcal N(0,1)$ as $\nu \to \infty$. For large $\nu$, the cdf of $\mathcal T_{\nu}$ can therefore be approximated by that of the gaussian. However, $\nu = 5$ in the problem. Unlike the normal case, deriving a lower bound for $P(X>a)$ isn't trivial. Consequently, the lower bounds we use are obtained by truncation of the true value. They could also be found by running simulations with $\delta_1$ only, but for the sake of simplicity we chose to truncate the true value instead.\newline
  The simulation method is identical to the normal case.
\end{enumerate}

{\centering
\begin{tabular}{|c||c|c|c|c|} \hline
$a$ &  $P(X>a)$ & Lower bound on $P(X>a)$& Mean of $(\delta_1-P(X>a))^2$ & Mean of $(\delta_2-P(X>a))^2$ \\ \hline
3 &
\begin{tabular}{c} $1.5049 \cdot 10^{-2}$
\end{tabular} &
\begin{tabular}{c} $1.5000 \cdot 10^{-2}$
\end{tabular} &
\begin{tabular}{c} $1.5019 \cdot 10^{-7}$
\end{tabular} &
\begin{tabular}{c} $1.4968 \cdot 10^{-7}$
\end{tabular}  \\ \hline
5 &
\begin{tabular}{c} $2.0523 \cdot 10^{-3}$
\end{tabular} &
\begin{tabular}{c} $2.0000 \cdot 10^{-3}$
\end{tabular} &
\begin{tabular}{c} $2.0805 \cdot 10^{-8}$
\end{tabular} &
\begin{tabular}{c} $2.0760 \cdot 10^{-8}$
\end{tabular}  \\ \hline
7&
\begin{tabular}{c} $4.5837 \cdot 10^{-4}$
\end{tabular} &
\begin{tabular}{c} $4.5000 \cdot 10^{-4}$
\end{tabular} &
\begin{tabular}{c} $4.6487 \cdot 10^{-9}$
\end{tabular} &
\begin{tabular}{c} $4.6454 \cdot 10^{-9}$
\end{tabular}  \\ \hline
\end{tabular}
\captionof{table}{Simulation results for $f\sim \mathcal T_{5}$}
}


\section*{Problem 5.9}
\begin{enumerate}[label=(\alph*)]
  \item Let $(x_1,z_1),\ldots,(x_n,z_n)$ be an i.i.d sample from $(X,Z)$. Since $Z$ follows a Bernoulli with parameter $\theta$, Bayes' theorem yields $f_{(X,Z)}(x_i,z_i) = f_{X|Z=z_i}(x_i)f_Z(z_i)$, hence $$\begin{aligned}
  L^c(\theta|\mathbf x, \mathbf z) &= \prod_{i=1}^n f_{(X,Z)}(x_i,z_i) = \prod_{i=1}^n  f_{X|Z=z_i}(x_i)f_Z(z_i)\\
  &= \prod_{i=1}^n (z_ig(x_i)+(1-z_i)h(x_i))\theta^{z_i} (1-\theta)^{1-z_i}
  \end{aligned}
  $$

  \item Applying Bayes' theorem again, $$\displaystyle f_{Z|X=x_i}(z_i) = \frac{f_{X|Z=z_i}(x_i)f_Z(z_i)}{f_X(x_i)} = \frac{(z_ig(x_i)+(1-z_i)h(x_i))\theta^{z_i} (1-\theta)^{1-z_i}}{\theta g(x_i) + (1-\theta)h(x_i)}$$ Since $z_i\in \{0,1\}$, the numerator can be rewritten as $[\theta g(x_i)]^{z_i}[(1-\theta) h(x_i)]^{1-z_i}$, and we identify the density of a Bernoulli with parameter $\displaystyle \frac{\theta g(x_i)}{\theta g(x_i) + (1-\theta)h(x_i)}$. The conditional distribution of $Z$ given $\theta$ and $X=x_i$ is thus $\mathcal B(\frac{\theta g(x_i)}{\theta g(x_i) + (1-\theta)h(x_i)})$, hence $$E(Z|\theta, X=x_i)= \frac{\theta g(x_i)}{\theta g(x_i) + (1-\theta)h(x_i)}$$
  Let $$\begin{aligned}
  Q(\theta|\hat \theta_j, \mathbf x) &= E_{\mathbf{z}|\hat \theta_j, \mathbf x}(\log L^c(\theta|\mathbf x, \mathbf z))\\
  &= E_{\mathbf{z}|\hat \theta_j, \mathbf x} \left(\log \prod_{i=1}^n [\theta g(x_i)]^{z_i}[(1-\theta) h(x_i)]^{1-z_i} \right)\\
  &= \sum_{i=1}^n E_{\mathbf{z}|\hat \theta_j, \mathbf x}\left( z_i \log(\theta g(x_i)) + (1-z_i)\log((1-\theta) h(x_i)) \right)\\
  &= \sum_{i=1}^n \log(\theta g(x_i)) \frac{\hat\theta_j g(x_i)}{\hat\theta_j g(x_i) + (1-\hat\theta_j)h(x_i)} + \log((1-\theta) h(x_i)) \frac{(1-\hat\theta_j)h(x_i)}{\hat\theta_j g(x_i) + (1-\hat\theta_j)h(x_i)}
  \end{aligned}
  $$
  Note that $\theta \mapsto Q(\theta|\hat \theta_j, \mathbf x)$ is differentiable and concave as the sum of concave functions. Critical points are therefore global maxima. Since $$\frac{d Q(\theta|\hat \theta_j, \mathbf x)}{d\theta} = \sum_{i=1}^n \frac{1}{\theta}\frac{\hat\theta_j g(x_i)}{\hat\theta_j g(x_i)+ (1-\hat\theta_j)h(x_i)} - \frac{1}{1-\theta}\frac{(1-\hat\theta_j)h(x_i)}{\hat\theta_j g(x_i) + (1-\hat\theta_j)h(x_i)}$$
  we have $\displaystyle \frac{d Q(\theta|\hat \theta_j, \mathbf x)}{d\theta} = 0 \iff (1-\theta)\sum_{i=1}^n\frac{\hat\theta_j g(x_i)}{\hat\theta_j g(x_i)+ (1-\hat\theta_j)h(x_i)} = \theta  \sum_{i=1}^n\frac{(1-\hat\theta_j)h(x_i)}{\hat\theta_j g(x_i) + (1-\hat\theta_j)h(x_i)}$
  which yields $$\hat \theta_{j+1} = \frac 1n \sum_{i=1}^n\frac{\hat\theta_j g(x_i)}{\hat\theta_j g(x_i)+ (1-\hat\theta_j)h(x_i)}$$

  \item Let $\phi:\theta \mapsto \frac 1n \sum_{i=1}^n\frac{\theta g(x_i)}{\theta g(x_i)+ (1-\theta)h(x_i)}$. We have proved that $\forall j, \hat \theta_{j+1} = \phi(\hat \theta_{j})$, so the EM sequence is a discrete dynamical system. Since $\phi'(\theta) = \sum_{i=1}^n \frac{g(x_i)h(x_i)}{(\theta g(x_i)+ (1-\theta)h(x_i))^2}\geq 0$, $\phi$ increases. Besides, it is bounded above by $1$. As a result, $\hat \theta_{j}$ is increasing and bounded, thus convergent.\newline \newline 
  $\hat \theta_{j}$ converges to some $\hat \theta$ such that $\phi(\hat \theta) = \hat \theta$, which rewrites as $$\begin{aligned}
    &\frac 1n \sum_{i=1}^n \frac{\hat \theta g(x_i)}{\hat \theta g(x_i)+ (1-\hat \theta)h(x_i)} = \hat \theta  = \frac 1n \sum_{i=1}^n \hat \theta\\
    \iff& \hat \theta(1-\hat \theta)\sum_{i=1}^n \frac{g(x_i)-h(x_i)}{\hat \theta g(x_i)+ (1-\hat \theta)h(x_i)} = 0 \\
  \end{aligned}$$
  We expect that $\hat \theta \notin \{0,1\}$ hence $\displaystyle \sum_{i=1}^n \frac{g(x_i)-h(x_i)}{\hat\theta g(x_i)+ (1-\hat\theta)h(x_i)} = 0$

  Note that the log-likelihood writes as $$\log L(\theta, \mathbf x) = \sum_{i=1}^n \log(\theta g(x_i)) + (1-\theta)h(x_i))$$ which is concave and differentiable. The first-order condition is $\displaystyle \sum_{i=1}^n \frac{g(x_i)-h(x_i)}{\theta g(x_i) + (1-\theta)h(x_i)} = 0$, hence $\hat \theta$ is a maximum likelihood estimator, as well as being the EM limiting estimator.
\end{enumerate}

\section*{Problem 6.9}
Let $P$ be the transition matrix of an aperiodic and irreducible chain on some finite state space $E=\{1,\ldots,p\}$. We will first prove the following lemma:\newline
\textbf{Lemma}: Let $A$ be a subset of $\mathbb N$ closed under addition and with $\gcd A = 1$. Then there exists $n_0$ such that $\forall n\geq n_0, n\in A$.\newline
\textit{Proof}: By Bézout's identity, there exists $a_1,\ldots, a_n\in A$ and $b_1,\ldots,b_n\in \mathbb Z$ such that $\sum_{k=1}^n a_kb_k =1$. Note that $$\underbrace{\sum_{k, b_k>0} a_k b_k}_{:=\;x\;\in A} = 1 + \underbrace{\sum_{k, b_k\leq0} a_k (-b_k)}_{:= \;y\;\in A}$$
If $y=0$, then $1=x\in A$ and we're done. Otherwise, $y\geq 1$. Let $m\geq y^2$ and write the Euclidean division of $m$ by $y$ as $m=ky+r$ with $0\leq r<y$. Note that $y^2\leq m < (k+1)y$, hence $k>y-1\implies k\geq y\implies k>r$. Besides, $m = ky+r = ky + r(x-y) = (k-r)y + rx$, where $k-r>0$, $x,y\in A$. Hence $m\in A$, and this holds for all $m\geq y^2$. \hfill $\square$\newline\newline
To apply the lemma, we prove that for any state $i$, the set $A_i:=\{n, P_{ii}^{(n)} >0\}$ is closed under addition. Indeed, if $P_{ii}^{(n)} >0$ and $P_{ii}^{(m)} >0$, by Chapman-Kolmogorov equation we have $P_{ii}^{(n+m)}\geq P_{ii}^{(n)}P_{ii}^{(m)}>0$. By the lemma, for every state $i$, there exists some $n_i$ such that $n\geq n_i\implies n\in A_i$. For fixed $i$ and any $j$, there exists by irreducibility some $m_{ij}$ such that $P_{ij}^{(m_{ij})} >0$. Thus, for $q\geq n_i + m_{ij}$, $P_{ij}^{(q)}\geq P_{ii}^{(q-m_{ij})} P_{ij}^{(m_{ij})}>0$, hence for $q\geq n_i + \max_{1\leq j\leq p} m_{ij}$ and any state $j$,  $P_{ij}^{(q)}>0$. Finally, it suffices to set $n_0:=\max_{1\leq i\leq p} (n_i + \max_{1\leq j\leq p} m_{ij})$ to get $$\forall n\geq n_0, \forall i,j, P_{ij}^{(n)}>0$$
Note that we proved a stronger statement than what was asked in the problem.\newline \newline
For the converse, if $P^{(n)}$ has positive coefficients, then all states communicate with one another, hence the chain is irreducible.


\section*{Problem 6.54}
\begin{enumerate}
  \item A two-state chain has transition matrix $P =\begin{pmatrix}
    a & 1-a\\
    1-b & b
  \end{pmatrix}$. If the chain is ergodic, is has a stationary distribution $\pi=(\pi_1,\pi_2)$ such that $\pi P = \pi$. After solving the resulting linear system of equations, one gets $\pi = (\frac{1-b}{2-a-b}, \frac{1-a}{2-a-b})$ and it's easy to check next that $\tilde P = P$, hence the chain is reversible.
  \item A chain with symmetric transition matrix has $(\frac 1n, \ldots, \frac 1n)$ as invariant distribution. Indeed, $$(\pi P)_j = \sum_{i=1}^n \pi_i P_{ij} = \frac 1n \sum_{i=1}^n P_{ij} = \frac 1n \sum_{i=1}^n P_{ji}= \frac 1n$$
  Therefore, $\tilde P_{ij} = \frac{\pi_j P_{ji}}{\pi_i}= P_{ji} = P_{ij}$ and $P$ is reversible.
  \item For the given matrix, solving the linear system $\pi P = \pi, \mathbf 1^T\pi = 1$ yields $\pi=(0.1,0.2,0.4,0.2,0.1)$. Note that $\tilde P_{12} = \frac{0.2\cdot 0.5}{0.1} = 1 \neq P_{12}= 0$, hence the chain is not reversible.
 \end{enumerate}


\section*{Appendix}
\begin{verbatim}
import numpy as np
from scipy.stats import norm, t 

#Simulations for the normal case
for a in [3, 5, 7]:
    beta = -4*1/np.sqrt(2*np.pi)*np.exp(-a*a/2)*(1/a-1/a**3)
    param_true = 1-norm.cdf(a)
    var1, var2 = [], []
    for _ in range(10000):
        sample = np.random.normal(size=100000)
        param_esti1 = np.mean(sample>a)
        param_esti2 = np.mean(sample>a) + beta*(np.mean(sample>0)-0.5)
        var1.append((param_true-param_esti1)**2)
        var2.append((param_true-param_esti2)**2)
    print('a='+str(a)+':', np.mean(var1), np.mean(var2))

#Simulations for the Student distribution
lower = [0.015, 0.002, 0.00045]
for (a, lower) in zip([3, 5, 7], lower):
    beta = -4*lower
    param_true = 1-t.cdf(a, df =5)
    var1, var2 = [], []
    for _ in range(10000):
        sample = np.random.standard_t(5, size=100000)
        param_esti1 = np.mean(sample>a)
        param_esti2 = np.mean(sample>a) + beta*(np.mean(sample>0)-0.5)
        var1.append((param_true-param_esti1)**2)
        var2.append((param_true-param_esti2)**2)
    print('a='+str(a)+':', np.mean(var1), np.mean(var2))
    
\end{verbatim}

\end{document}